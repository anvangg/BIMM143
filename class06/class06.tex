% Options for packages loaded elsewhere
\PassOptionsToPackage{unicode}{hyperref}
\PassOptionsToPackage{hyphens}{url}
%
\documentclass[
]{article}
\usepackage{lmodern}
\usepackage{amssymb,amsmath}
\usepackage{ifxetex,ifluatex}
\ifnum 0\ifxetex 1\fi\ifluatex 1\fi=0 % if pdftex
  \usepackage[T1]{fontenc}
  \usepackage[utf8]{inputenc}
  \usepackage{textcomp} % provide euro and other symbols
\else % if luatex or xetex
  \usepackage{unicode-math}
  \defaultfontfeatures{Scale=MatchLowercase}
  \defaultfontfeatures[\rmfamily]{Ligatures=TeX,Scale=1}
\fi
% Use upquote if available, for straight quotes in verbatim environments
\IfFileExists{upquote.sty}{\usepackage{upquote}}{}
\IfFileExists{microtype.sty}{% use microtype if available
  \usepackage[]{microtype}
  \UseMicrotypeSet[protrusion]{basicmath} % disable protrusion for tt fonts
}{}
\makeatletter
\@ifundefined{KOMAClassName}{% if non-KOMA class
  \IfFileExists{parskip.sty}{%
    \usepackage{parskip}
  }{% else
    \setlength{\parindent}{0pt}
    \setlength{\parskip}{6pt plus 2pt minus 1pt}}
}{% if KOMA class
  \KOMAoptions{parskip=half}}
\makeatother
\usepackage{xcolor}
\IfFileExists{xurl.sty}{\usepackage{xurl}}{} % add URL line breaks if available
\IfFileExists{bookmark.sty}{\usepackage{bookmark}}{\usepackage{hyperref}}
\hypersetup{
  pdftitle={class06Rfunctions},
  pdfauthor={Ashley Vang},
  hidelinks,
  pdfcreator={LaTeX via pandoc}}
\urlstyle{same} % disable monospaced font for URLs
\usepackage[margin=1in]{geometry}
\usepackage{color}
\usepackage{fancyvrb}
\newcommand{\VerbBar}{|}
\newcommand{\VERB}{\Verb[commandchars=\\\{\}]}
\DefineVerbatimEnvironment{Highlighting}{Verbatim}{commandchars=\\\{\}}
% Add ',fontsize=\small' for more characters per line
\usepackage{framed}
\definecolor{shadecolor}{RGB}{248,248,248}
\newenvironment{Shaded}{\begin{snugshade}}{\end{snugshade}}
\newcommand{\AlertTok}[1]{\textcolor[rgb]{0.94,0.16,0.16}{#1}}
\newcommand{\AnnotationTok}[1]{\textcolor[rgb]{0.56,0.35,0.01}{\textbf{\textit{#1}}}}
\newcommand{\AttributeTok}[1]{\textcolor[rgb]{0.77,0.63,0.00}{#1}}
\newcommand{\BaseNTok}[1]{\textcolor[rgb]{0.00,0.00,0.81}{#1}}
\newcommand{\BuiltInTok}[1]{#1}
\newcommand{\CharTok}[1]{\textcolor[rgb]{0.31,0.60,0.02}{#1}}
\newcommand{\CommentTok}[1]{\textcolor[rgb]{0.56,0.35,0.01}{\textit{#1}}}
\newcommand{\CommentVarTok}[1]{\textcolor[rgb]{0.56,0.35,0.01}{\textbf{\textit{#1}}}}
\newcommand{\ConstantTok}[1]{\textcolor[rgb]{0.00,0.00,0.00}{#1}}
\newcommand{\ControlFlowTok}[1]{\textcolor[rgb]{0.13,0.29,0.53}{\textbf{#1}}}
\newcommand{\DataTypeTok}[1]{\textcolor[rgb]{0.13,0.29,0.53}{#1}}
\newcommand{\DecValTok}[1]{\textcolor[rgb]{0.00,0.00,0.81}{#1}}
\newcommand{\DocumentationTok}[1]{\textcolor[rgb]{0.56,0.35,0.01}{\textbf{\textit{#1}}}}
\newcommand{\ErrorTok}[1]{\textcolor[rgb]{0.64,0.00,0.00}{\textbf{#1}}}
\newcommand{\ExtensionTok}[1]{#1}
\newcommand{\FloatTok}[1]{\textcolor[rgb]{0.00,0.00,0.81}{#1}}
\newcommand{\FunctionTok}[1]{\textcolor[rgb]{0.00,0.00,0.00}{#1}}
\newcommand{\ImportTok}[1]{#1}
\newcommand{\InformationTok}[1]{\textcolor[rgb]{0.56,0.35,0.01}{\textbf{\textit{#1}}}}
\newcommand{\KeywordTok}[1]{\textcolor[rgb]{0.13,0.29,0.53}{\textbf{#1}}}
\newcommand{\NormalTok}[1]{#1}
\newcommand{\OperatorTok}[1]{\textcolor[rgb]{0.81,0.36,0.00}{\textbf{#1}}}
\newcommand{\OtherTok}[1]{\textcolor[rgb]{0.56,0.35,0.01}{#1}}
\newcommand{\PreprocessorTok}[1]{\textcolor[rgb]{0.56,0.35,0.01}{\textit{#1}}}
\newcommand{\RegionMarkerTok}[1]{#1}
\newcommand{\SpecialCharTok}[1]{\textcolor[rgb]{0.00,0.00,0.00}{#1}}
\newcommand{\SpecialStringTok}[1]{\textcolor[rgb]{0.31,0.60,0.02}{#1}}
\newcommand{\StringTok}[1]{\textcolor[rgb]{0.31,0.60,0.02}{#1}}
\newcommand{\VariableTok}[1]{\textcolor[rgb]{0.00,0.00,0.00}{#1}}
\newcommand{\VerbatimStringTok}[1]{\textcolor[rgb]{0.31,0.60,0.02}{#1}}
\newcommand{\WarningTok}[1]{\textcolor[rgb]{0.56,0.35,0.01}{\textbf{\textit{#1}}}}
\usepackage{graphicx,grffile}
\makeatletter
\def\maxwidth{\ifdim\Gin@nat@width>\linewidth\linewidth\else\Gin@nat@width\fi}
\def\maxheight{\ifdim\Gin@nat@height>\textheight\textheight\else\Gin@nat@height\fi}
\makeatother
% Scale images if necessary, so that they will not overflow the page
% margins by default, and it is still possible to overwrite the defaults
% using explicit options in \includegraphics[width, height, ...]{}
\setkeys{Gin}{width=\maxwidth,height=\maxheight,keepaspectratio}
% Set default figure placement to htbp
\makeatletter
\def\fps@figure{htbp}
\makeatother
\setlength{\emergencystretch}{3em} % prevent overfull lines
\providecommand{\tightlist}{%
  \setlength{\itemsep}{0pt}\setlength{\parskip}{0pt}}
\setcounter{secnumdepth}{-\maxdimen} % remove section numbering

\title{class06Rfunctions}
\author{Ashley Vang}
\date{10/13/2022}

\begin{document}
\maketitle

\#Function basics

All functions in R consist of at least 3 things:

-A \textbf{name} (we can pick this but it must start with a character)
-Input \textbf{arguments} (there can be multiple comma separated inputs)
-The \textbf{body }(where work actually happens)

\begin{Shaded}
\begin{Highlighting}[]
\CommentTok{# Example input vectors to start with}
\NormalTok{student1 <-}\StringTok{ }\KeywordTok{c}\NormalTok{(}\DecValTok{100}\NormalTok{, }\DecValTok{100}\NormalTok{, }\DecValTok{100}\NormalTok{, }\DecValTok{100}\NormalTok{, }\DecValTok{100}\NormalTok{, }\DecValTok{100}\NormalTok{, }\DecValTok{100}\NormalTok{, }\DecValTok{90}\NormalTok{)}
\NormalTok{student2 <-}\StringTok{ }\KeywordTok{c}\NormalTok{(}\DecValTok{100}\NormalTok{, }\OtherTok{NA}\NormalTok{, }\DecValTok{90}\NormalTok{, }\DecValTok{90}\NormalTok{, }\DecValTok{90}\NormalTok{, }\DecValTok{90}\NormalTok{, }\DecValTok{97}\NormalTok{, }\DecValTok{80}\NormalTok{)}
\NormalTok{student3 <-}\StringTok{ }\KeywordTok{c}\NormalTok{(}\DecValTok{90}\NormalTok{, }\OtherTok{NA}\NormalTok{, }\OtherTok{NA}\NormalTok{, }\OtherTok{NA}\NormalTok{, }\OtherTok{NA}\NormalTok{, }\OtherTok{NA}\NormalTok{, }\OtherTok{NA}\NormalTok{, }\OtherTok{NA}\NormalTok{)}
\end{Highlighting}
\end{Shaded}

I can start by using the `mean()' function to calculate an average.

\begin{Shaded}
\begin{Highlighting}[]
\KeywordTok{mean}\NormalTok{(student1)}
\end{Highlighting}
\end{Shaded}

\begin{verbatim}
## [1] 98.75
\end{verbatim}

I found the `min()' function to find the minimum value in a vector.

\begin{Shaded}
\begin{Highlighting}[]
\KeywordTok{min}\NormalTok{(student1)}
\end{Highlighting}
\end{Shaded}

\begin{verbatim}
## [1] 90
\end{verbatim}

Looking at the ``See Also'' section of the `min()' help page I found out
about `which.min()'

\begin{Shaded}
\begin{Highlighting}[]
\KeywordTok{which.min}\NormalTok{(student1)}
\end{Highlighting}
\end{Shaded}

\begin{verbatim}
## [1] 8
\end{verbatim}

\begin{Shaded}
\begin{Highlighting}[]
\NormalTok{student1}
\end{Highlighting}
\end{Shaded}

\begin{verbatim}
## [1] 100 100 100 100 100 100 100  90
\end{verbatim}

\begin{Shaded}
\begin{Highlighting}[]
\NormalTok{student1[}\DecValTok{1}\OperatorTok{:}\DecValTok{7}\NormalTok{]}
\end{Highlighting}
\end{Shaded}

\begin{verbatim}
## [1] 100 100 100 100 100 100 100
\end{verbatim}

I can get the same vector without the 8th element with the minus index
trick\ldots{}

\begin{Shaded}
\begin{Highlighting}[]
\NormalTok{student1[}\OperatorTok{-}\DecValTok{8}\NormalTok{]}
\end{Highlighting}
\end{Shaded}

\begin{verbatim}
## [1] 100 100 100 100 100 100 100
\end{verbatim}

So I will combine the output of `which.min()' with the minus index trick
to get the student scores without the lowest value.

\begin{Shaded}
\begin{Highlighting}[]
\KeywordTok{mean}\NormalTok{(student1[}\OperatorTok{-}\KeywordTok{which.min}\NormalTok{(student1)])}
\end{Highlighting}
\end{Shaded}

\begin{verbatim}
## [1] 100
\end{verbatim}

Hmm\ldots{} For student2 this gives NA

\begin{Shaded}
\begin{Highlighting}[]
\KeywordTok{mean}\NormalTok{(student2[}\OperatorTok{-}\KeywordTok{which.min}\NormalTok{(student2)])}
\end{Highlighting}
\end{Shaded}

\begin{verbatim}
## [1] NA
\end{verbatim}

I see there is an `na.rm=FALSE' by default argument to the `mean()'
function. Will this help us?

\begin{Shaded}
\begin{Highlighting}[]
\KeywordTok{mean}\NormalTok{(student2[}\OperatorTok{-}\KeywordTok{which.min}\NormalTok{(student2)], }\DataTypeTok{na.rm=}\OtherTok{TRUE}\NormalTok{)}
\end{Highlighting}
\end{Shaded}

\begin{verbatim}
## [1] 92.83333
\end{verbatim}

\begin{Shaded}
\begin{Highlighting}[]
\KeywordTok{mean}\NormalTok{(student3[}\OperatorTok{-}\KeywordTok{which.min}\NormalTok{(student3)], }\DataTypeTok{na.rm=}\OtherTok{TRUE}\NormalTok{)}
\end{Highlighting}
\end{Shaded}

\begin{verbatim}
## [1] NaN
\end{verbatim}

We need another way\ldots{}

How about we replace all NA (missing values) with zero.

\begin{Shaded}
\begin{Highlighting}[]
\NormalTok{student3}
\end{Highlighting}
\end{Shaded}

\begin{verbatim}
## [1] 90 NA NA NA NA NA NA NA
\end{verbatim}

\begin{Shaded}
\begin{Highlighting}[]
\KeywordTok{is.na}\NormalTok{(student3)}
\end{Highlighting}
\end{Shaded}

\begin{verbatim}
## [1] FALSE  TRUE  TRUE  TRUE  TRUE  TRUE  TRUE  TRUE
\end{verbatim}

\begin{Shaded}
\begin{Highlighting}[]
\NormalTok{student3 [}\KeywordTok{is.na}\NormalTok{(student3)] <-}\StringTok{ }\DecValTok{0}
\NormalTok{student3}
\end{Highlighting}
\end{Shaded}

\begin{verbatim}
## [1] 90  0  0  0  0  0  0  0
\end{verbatim}

\begin{Shaded}
\begin{Highlighting}[]
\KeywordTok{mean}\NormalTok{(student3[}\OperatorTok{-}\KeywordTok{which.min}\NormalTok{(student3)])}
\end{Highlighting}
\end{Shaded}

\begin{verbatim}
## [1] 12.85714
\end{verbatim}

All this copy paste is silly and dangerous- time to write a function.

\begin{Shaded}
\begin{Highlighting}[]
\NormalTok{x<-student3}
\NormalTok{x [}\KeywordTok{is.na}\NormalTok{(x)] <-}\StringTok{ }\DecValTok{0}
\KeywordTok{mean}\NormalTok{(x [}\OperatorTok{-}\KeywordTok{which.min}\NormalTok{(x)])}
\end{Highlighting}
\end{Shaded}

\begin{verbatim}
## [1] 12.85714
\end{verbatim}

I now have my working snippet of code that I have simplified to work
with any student `x'.

\begin{Shaded}
\begin{Highlighting}[]
\NormalTok{x [}\KeywordTok{is.na}\NormalTok{(x)] <-}\StringTok{ }\DecValTok{0}
\KeywordTok{mean}\NormalTok{(x [}\OperatorTok{-}\KeywordTok{which.min}\NormalTok{(x)])}
\end{Highlighting}
\end{Shaded}

\begin{verbatim}
## [1] 12.85714
\end{verbatim}

\hypertarget{function-basics}{%
\subsection{Function basics}\label{function-basics}}

Now turn into a function:

\begin{Shaded}
\begin{Highlighting}[]
\NormalTok{grade <-}\StringTok{ }\ControlFlowTok{function}\NormalTok{(x)\{}
\NormalTok{  x [}\KeywordTok{is.na}\NormalTok{(x)] <-}\StringTok{ }\DecValTok{0}
  \KeywordTok{mean}\NormalTok{(x [}\OperatorTok{-}\KeywordTok{which.min}\NormalTok{(x)])}
\NormalTok{\}}
\end{Highlighting}
\end{Shaded}

\begin{Shaded}
\begin{Highlighting}[]
\KeywordTok{grade}\NormalTok{(student1)}
\end{Highlighting}
\end{Shaded}

\begin{verbatim}
## [1] 100
\end{verbatim}

\hypertarget{q2}{%
\subsection{Q2}\label{q2}}

\begin{quote}
Q2. Using your grade() function and the supplied gradebook, Who is the
top scoring student overall in the gradebook?
\end{quote}

\begin{Shaded}
\begin{Highlighting}[]
\NormalTok{url <-}\StringTok{"https://tinyurl.com/gradeinput"}
\NormalTok{gradebook <-}\KeywordTok{read.csv}\NormalTok{(url, }\DataTypeTok{row.names=}\DecValTok{1}\NormalTok{)}
\end{Highlighting}
\end{Shaded}

Have a look at the first 6 rows\ldots{}

\begin{Shaded}
\begin{Highlighting}[]
\KeywordTok{head}\NormalTok{(gradebook)}
\end{Highlighting}
\end{Shaded}

\begin{verbatim}
##           hw1 hw2 hw3 hw4 hw5
## student-1 100  73 100  88  79
## student-2  85  64  78  89  78
## student-3  83  69  77 100  77
## student-4  88  NA  73 100  76
## student-5  88 100  75  86  79
## student-6  89  78 100  89  77
\end{verbatim}

Time to learn about the `apply()' function.

\begin{Shaded}
\begin{Highlighting}[]
\NormalTok{results <-}\KeywordTok{apply}\NormalTok{(gradebook,}\DecValTok{1}\NormalTok{,grade)}
\end{Highlighting}
\end{Shaded}

Which student did the best overall?

\begin{Shaded}
\begin{Highlighting}[]
\KeywordTok{which.max}\NormalTok{(results)}
\end{Highlighting}
\end{Shaded}

\begin{verbatim}
## student-18 
##         18
\end{verbatim}

\begin{Shaded}
\begin{Highlighting}[]
\NormalTok{results[}\KeywordTok{which.max}\NormalTok{(results)]}
\end{Highlighting}
\end{Shaded}

\begin{verbatim}
## student-18 
##       94.5
\end{verbatim}

\hypertarget{q3}{%
\subsection{Q3}\label{q3}}

\begin{quote}
Q3 From your analysis of the gradebook, which homework was toughest on
students (i.e.~obtained the lowest scores overall?
\end{quote}

\begin{Shaded}
\begin{Highlighting}[]
\KeywordTok{which.min}\NormalTok{(}\KeywordTok{apply}\NormalTok{(gradebook, }\DecValTok{2}\NormalTok{, sum, }\DataTypeTok{na.rm=}\OtherTok{TRUE}\NormalTok{))}
\end{Highlighting}
\end{Shaded}

\begin{verbatim}
## hw2 
##   2
\end{verbatim}

\hypertarget{q4}{%
\subsection{Q4}\label{q4}}

\begin{quote}
Q4 From your analysis of the gradebook, which homework was most
predictive of overall score (i.e.~highest correlation with average grade
score)?
\end{quote}

\begin{Shaded}
\begin{Highlighting}[]
\NormalTok{mask <-}\StringTok{ }\NormalTok{gradebook}
\NormalTok{mask [}\KeywordTok{is.na}\NormalTok{(mask)] <-}\StringTok{ }\DecValTok{0}

\KeywordTok{cor}\NormalTok{(mask}\OperatorTok{$}\NormalTok{hw5, results)}
\end{Highlighting}
\end{Shaded}

\begin{verbatim}
## [1] 0.6325982
\end{verbatim}

\begin{Shaded}
\begin{Highlighting}[]
\KeywordTok{cor}\NormalTok{(mask}\OperatorTok{$}\NormalTok{hw1, results)}
\end{Highlighting}
\end{Shaded}

\begin{verbatim}
## [1] 0.4250204
\end{verbatim}

Or use apply\ldots{}

\begin{Shaded}
\begin{Highlighting}[]
\KeywordTok{apply}\NormalTok{(mask, }\DecValTok{2}\NormalTok{,cor, }\DataTypeTok{y=}\NormalTok{results)}
\end{Highlighting}
\end{Shaded}

\begin{verbatim}
##       hw1       hw2       hw3       hw4       hw5 
## 0.4250204 0.1767780 0.3042561 0.3810884 0.6325982
\end{verbatim}

\end{document}
